% This file has been generated from the lua sources using LuaDoc.
% To regenerate it call "make genluadoc" in
% doc/generic/pgf/version-for-luatex/en.

\paragraph{pgflibrarygraphdrawing-helper.lua}


\begin{luacommand}{{copyTable}(\meta{table},\meta{result})}
Copies a table, preserving its metatable.

Parameters:
\begin{itemize}
	\item[] \meta{table} \subitem The table from which values are copied.\item[] \meta{result} \subitem The table to which values are copied or nil.
\end{itemize}


Return value:
\begin{itemize} \item[] A new table containing all the keys and values. \end{itemize}


\end{luacommand}\begin{luacommand}{{countKeys}(\meta{table})}
Counts keys in an dictionary, where value is nil.

Parameters:
\begin{itemize}
	\item[] \meta{table} \subitem Dictionary.
\end{itemize}


Return value:
\begin{itemize} \item[] Number of keys. \end{itemize}


\end{luacommand}\begin{luacommand}{{filter}(\meta{iterator},\meta{test})}
Returns all results from iterator for which test returns a true value.



\end{luacommand}\begin{luacommand}{{findTable}(\meta{table},\meta{object})}
Finds an object in a table.


Return value:
\begin{itemize} \item[] The first index for a value which is equal to the object or nil. \end{itemize}


\end{luacommand}\begin{luacommand}{{keys}(\meta{map})}
Returns all keys in arbitrary order.



See also:
\begin{itemize}
	\item[] |pairs|
\end{itemize}

\end{luacommand}\begin{luacommand}{{mergeTable}(\meta{values},\meta{defaults})}
Merges two tables. Every nil value in values is replaced by its default value in defaults.  The metatable from defaults is likewise preserved. As luatex supplies its own version of table.merge, we can't use that same name.

Parameters:
\begin{itemize}
	\item[] \meta{values} \subitem New values or nil.  Same as return value if non-nil.
\end{itemize}



\end{luacommand}\begin{luacommand}{{parseBraces}(\meta{string},\meta{default})}
Parses a braced list of {key}{value} pairs and returns a table mapping keys to values.



\end{luacommand}\begin{luacommand}{{values}(\meta{table})}
Returns all values in numerical order.



See also:
\begin{itemize}
	\item[] |ipairs|
\end{itemize}

\end{luacommand}
