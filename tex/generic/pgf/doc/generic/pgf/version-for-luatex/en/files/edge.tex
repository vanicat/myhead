% This file has been generated from the lua sources using LuaDoc.
% To regenerate it call "make genluadoc" in
% doc/generic/pgf/version-for-luatex/en.

\paragraph{pgflibrarygraphdrawing-edge.lua}


\begin{luacommand}{{Edge:\textunderscore{}\textunderscore{}tostring}()}
Returns a readable string representation of the edge.


Return value:
\begin{itemize} \item[] String representation of the edge. \end{itemize}


\end{luacommand}\begin{luacommand}{{Edge:addNode}(\meta{node})}
Adds node to the edge.

Parameters:
\begin{itemize}
	\item[] \meta{node} \subitem The node to be added to the edge.
\end{itemize}



\end{luacommand}\begin{luacommand}{{Edge:containsNode}(\meta{node})}
Tests if edge contains a node.


Return value:
\begin{itemize} \item[] True if the edge contains a node. \end{itemize}


\end{luacommand}\begin{luacommand}{{Edge:copy}()}
Copies an edge (preventing accidental use).


Return value:
\begin{itemize} \item[] Shallow copy of the edge. \end{itemize}


\end{luacommand}\begin{luacommand}{{Edge:getDegree}()}
Returns number of nodes on the edge.


Return value:
\begin{itemize} \item[] Number of nodes of the edge. \end{itemize}


\end{luacommand}\begin{luacommand}{{Edge:getNeighbour}(\meta{node})}
Gets first neighbour of the node (disregarding hyperedges).

Parameters:
\begin{itemize}
	\item[] \meta{node} \subitem The node which first neighbour should be returned.
\end{itemize}


Return value:
\begin{itemize} \item[] The first neighbour of the node. \end{itemize}


\end{luacommand}\begin{luacommand}{{Edge:getNeighbours}(\meta{node})}
Returns all neighbours of a node.

Parameters:
\begin{itemize}
	\item[] \meta{node} \subitem The node which neighbours should be returned.
\end{itemize}


Return value:
\begin{itemize} \item[] Array of neighbour nodes. \end{itemize}


\end{luacommand}\begin{luacommand}{{Edge:getNodes}()}
Returns the nodes of an edge.


Return value:
\begin{itemize} \item[] Array of nodes of the edge. \end{itemize}


\end{luacommand}\begin{luacommand}{{Edge:getPath}()}
Returns the path of an edge.


Return value:
\begin{itemize} \item[] The path the edge belongs to. \end{itemize}


\end{luacommand}\begin{luacommand}{{Edge:isHyperedge}()}
Returns a boolean whether the edge is a hyperedge.


Return value:
\begin{itemize} \item[] True if the edge is a hyperedge. \end{itemize}


\end{luacommand}\begin{luacommand}{{Edge:new}(\meta{values})}
Creates an edge between nodes of a graph.

Parameters:
\begin{itemize}
	\item[] \meta{values} \subitem Values (e.g. direction) to be merged with the default-metatable of an edge.
\end{itemize}


Return value:
\begin{itemize} \item[] The new edge. \end{itemize}


\end{luacommand}\begin{luacommand}{{Edge:setPath}(\meta{path})}
Sets the path of an edge.

Parameters:
\begin{itemize}
	\item[] \meta{path} \subitem The path the edge belongs to.
\end{itemize}



\end{luacommand}
