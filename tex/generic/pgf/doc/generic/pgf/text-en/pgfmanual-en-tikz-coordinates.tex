% Copyright 2006 by Till Tantau
%
% This file may be distributed and/or modified
%
% 1. under the LaTeX Project Public License and/or
% 2. under the GNU Free Documentation License.
%
% See the file doc/generic/pgf/licenses/LICENSE for more details.

\section{Specifying Coordinates}


\subsection{Overview}

A \emph{coordinate} is a position on the canvas on which your picture
is drawn. \tikzname\ uses a special syntax for specifying
coordinates. Coordinates are always put in round brackets. The general
syntax is
\declare{|(|\opt{|[|\meta{options}|]|}\meta{coordinate  specification}|)|}.

The \meta{coordinate specification} specifies coordinates using one of
many different possible \emph{coordinate systems}. Examples are the
Cartesian coordinate system or polar coordinates or spherical
coordinates. No matter which coordinate system is used, in the end, a
specific point on the canvas is represented by the coordinate.

There are two ways of specifying which coordinate system should be used:
\begin{description}
\item[Explicitly] You can specify the coordinate system explicitly. To
  do so, you give the name of the coordinate system at the beginning,
  followed by |cs:|, which stands for ``coordinate system,'' followed
  by a specification of the coordinate using the key-value
  syntax. Thus, the general syntax for \meta{coordinate specification}
  in the explicit case is |(|\meta{coordinate system}| cs:|\meta{list
    of key-value pairs specific to the coordinate system}|)|.
\item[Implicitly] The explicit specification is often too verbose when
  numerous coordinates should be given. Because of this, for the
  coordinate systems that you are likely to use often a special syntax
  is provided. \tikzname\ will notice when you use a coordinate
  specified in a special syntax and will choose the correct coordinate
  system automatically.
\end{description}

Here is an example in which explicit the coordinate systems are
specified explicitly:
\begin{codeexample}[]
\begin{tikzpicture}
  \draw[help lines] (0,0) grid (3,2);
  \draw (canvas cs:x=0cm,y=2mm)
     -- (canvas polar cs:radius=2cm,angle=30);
\end{tikzpicture}
\end{codeexample}
In the next example, the coordinate systems are implicit:
\begin{codeexample}[]
\begin{tikzpicture}
  \draw[help lines] (0,0) grid (3,2);
  \draw (0cm,2mm) -- (30:2cm);
\end{tikzpicture}
\end{codeexample}

It is possible to give options that apply only to a single
coordinate, although this makes sense for transformation options
only. To give transformation options for a single coordinate, give
these options at the beginning in brackets:
\begin{codeexample}[]
\begin{tikzpicture}
  \draw[help lines] (0,0) grid (3,2);
  \draw      (0,0) -- (1,1);
  \draw[red] (0,0) -- ([xshift=3pt] 1,1);
  \draw      (1,0) -- +(30:2cm);
  \draw[red] (1,0) -- +([shift=(135:5pt)] 30:2cm);
\end{tikzpicture}
\end{codeexample}


\subsection{Coordinate Systems}

\subsubsection{Canvas, XYZ, and Polar Coordinate Systems}

Let us start with the basic coordinate systems.

\begin{coordinatesystem}{canvas}
  The simplest way of specifying a coordinate is to use the |canvas|
  coordinate system. You provide a dimension $d_x$ using the |x=|
  option and another dimension $d_y$ using the |y=| option. The position on
  the canvas is located at the position that is $d_x$ to the right and
  $d_y$ above the origin.

  \begin{key}{/tikz/cs/x=\meta{dimension} (initially 0pt)}
    Distance by which the coordinate
    is to the right of the origin. You can also write things like
    |1cm+2pt| since the mathematical engine is used to evaluate the
    \meta{dimension}.
  \end{key}

  \begin{key}{/tikz/cs/y=\meta{dimension} (initially 0pt)}
    Distance by which the coordinate
    is above the origin.
  \end{key}

\begin{codeexample}[]
\begin{tikzpicture}
  \draw[help lines] (0,0) grid (3,2);

  \fill (canvas cs:x=1cm,y=1.5cm)    circle (2pt);
  \fill (canvas cs:x=2cm,y=-5mm+2pt) circle (2pt);
\end{tikzpicture}
\end{codeexample}

  To specify a coordinate in the coordinate system implicitly, you use
  two dimensions that are separated by a comma as in |(0cm,3pt)| or
  |(2cm,\textheight)|.
\begin{codeexample}[]
\begin{tikzpicture}
  \draw[help lines] (0,0) grid (3,2);

  \fill (1cm,1.5cm)    circle (2pt);
  \fill (2cm,-5mm+2pt) circle (2pt);
\end{tikzpicture}
\end{codeexample}
\end{coordinatesystem}


\begin{coordinatesystem}{xyz}
  The |xyz| coordinate system allows you to specify a point as a
  multiple of three vectors called the $x$-, $y$-, and
  $z$-vectors.  By default, the $x$-vector points 1cm to the right,
  the $y$-vector points 1cm upwards, but this can be changed
  arbitrarily as explained in Section~\ref{section-xyz}. The default
  $z$-vector points to $\bigl(-3.85\textrm{mm},-3.85\textrm{mm}\bigr)$.

  To specify the factors by which the vectors should be multiplied
  before being added, you use the following three options:
  \begin{key}{/tikz/cs/x=\meta{factor} (initially 0)}
    Factor by which the $x$-vector is multiplied.
  \end{key}
  \begin{key}{/tikz/cs/y=\meta{factor} (initially 0)}
    Works like |x|.
  \end{key}
  \begin{key}{/tikz/cs/z=\meta{factor} (initially 0)}
    Works like |x|.
  \end{key}

\begin{codeexample}[]
\begin{tikzpicture}[->]
  \draw (0,0) -- (xyz cs:x=1);
  \draw (0,0) -- (xyz cs:y=1);
  \draw (0,0) -- (xyz cs:z=1);
\end{tikzpicture}
\end{codeexample}

  This coordinate system can also be selected implicitly. To do so,
  you just provide two or three comma-separated factors (not
  dimensions).
\begin{codeexample}[]
\begin{tikzpicture}[->]
  \draw (0,0) -- (1,0);
  \draw (0,0) -- (0,1,0);
  \draw (0,0) -- (0,0,1);
\end{tikzpicture}
\end{codeexample}
\end{coordinatesystem}

\emph{Note:} It is possible to use coordinates like |(1,2cm)|, which
are neither |canvas| coordinates nor |xyz| coordinates. The rule is
the following: If a coordinate is of the implicit form
|(|\meta{x}|,|\meta{y}|)|, then \meta{x} and \meta{y} are checked,
independently, whether they have a dimension or whether they are
dimensionless. If both have a dimension, the |canvas| coordinate
system is used. If both lack a dimension, the |xyz| coordinate system
is used. If \meta{x} has a dimension and \meta{y} has not, then the
sum of two coordinate |(|\meta{x}|,0pt)| and |(0,|\meta{y}|)| is
used. If \meta{y} has a dimension and \meta{x} has not, then the sum
of two coordinate |(|\meta{x}|,0)| and |(0pt,|\meta{y}|)| is used.

\emph{Note furthermore:} An expression like |(2+3cm,0)| does
\emph{not} mean the same as |(2cm+3cm,0)|. Instead, if \meta{x} or
\meta{y} internally uses a mixture of dimensions and dimensionless
values, then all dimensionless values are ``upgraded'' to dimensions
by interpreting them as |pt|. So, |2+3cm| is the same dimension as
|2pt+3cm|.

\begin{coordinatesystem}{canvas polar}
  The |canvas polar| coordinate system allows you to specify
  polar coordinates. You provide an angle using the |angle=| option
  and a radius using the |radius=| option. This yields the point on
  the canvas that is at the given radius distance from the origin at
  the given degree. An angle of zero degrees to the right, a degree of
  90 upward.
  \begin{key}{/tikz/cs/angle=\meta{degrees}}
    The angle of the coordinate.
    The angle must always be given in degrees and should be between
    $-360$ and $720$.
  \end{key}
  \begin{key}{/tikz/cs/radius=\meta{dimension}}
    The distance from the origin.
  \end{key}
  \begin{key}{/tikz/cs/x radius=\meta{dimension}}
    A polar coordinate is,
    after all, just a point on a circle of the given \meta{radius}. When
    you provide an $x$-radius and also a $y$-radius, you specify an
    ellipse instead of a circle. The |radius| option has the same effect
    as specifying identical |x radius| and |y radius| options.
  \end{key}
  \begin{key}{/tikz/cs/y radius=\meta{dimension}}
    Works like |x radius|.
  \end{key}
\begin{codeexample}[]
\tikz \draw (0,0) -- (canvas polar cs:angle=30,radius=1cm);
\end{codeexample}

  The implicit form for canvas polar coordinates is the following:
  you specify the angle and the distance, separated by a colon as in
  |(30:1cm)|.

\begin{codeexample}[]
\tikz \draw    (0cm,0cm) -- (30:1cm) -- (60:1cm) -- (90:1cm)
            -- (120:1cm) -- (150:1cm) -- (180:1cm);
\end{codeexample}

  Two different radii are specified by writing |(30:1cm and 2cm)|.

  For the implicit form, instead of an angle given as a number you can
  also use certain words. For example, |up| is the same as |90|, so
  that you can write |\tikz \draw (0,0) -- (2ex,0pt) -- +(up:1ex);|
  and get \tikz \draw (0,0) -- (2ex,0pt) -- +(up:1ex);. Apart from |up|
  you can use |down|, |left|, |right|, |north|, |south|, |west|, |east|,
  |north east|, |north west|, |south east|, |south west|, all of which
  have their natural meaning.
\end{coordinatesystem}

\begin{coordinatesystem}{xyz polar}
  This coordinate system work similarly to the |canvas polar|
  system. However, the radius and the angle are interpreted in the
  $xy$-coordinate system, not in the canvas system. More detailed,
  consider the circle or ellipse whose half axes are given by the
  current $x$-vector and the current $y$-vector. Then, consider the
  point that lies at a given angle on this ellipse, where an angle of
  zero is the same as the $x$-vector and an angle of 90 is the
  $y$-vector. Finally, multiply the resulting vector by the given
  radius factor. Voil\`a.
  \begin{key}{/tikz/cs/angle=\meta{degrees}}
    The angle of the coordinate
    interpreted in the ellipse whose axes are the $x$-vector and the
    $y$-vector.
  \end{key}
  \begin{key}{/tikz/cs/radius=\meta{factor}}
    A factor by which the $x$-vector
    and $y$-vector are multiplied prior to forming the ellipse.
  \end{key}
  \begin{key}{/tikz/cs/x radius=\meta{dimension}} A specific factor by
    which only the $x$-vector is multiplied.
  \end{key}
  \begin{key}{/tikz/cs/y radius=\meta{dimension}}
    Works like |x radius|.
  \end{key}
\begin{codeexample}[]
\begin{tikzpicture}[x=1.5cm,y=1cm]
  \draw[help lines] (0cm,0cm) grid (3cm,2cm);

  \draw (0,0) -- (xyz polar cs:angle=0,radius=1);
  \draw (0,0) -- (xyz polar cs:angle=30,radius=1);
  \draw (0,0) -- (xyz polar cs:angle=60,radius=1);
  \draw (0,0) -- (xyz polar cs:angle=90,radius=1);

  \draw (xyz polar cs:angle=0,radius=2)
     -- (xyz polar cs:angle=30,radius=2)
     -- (xyz polar cs:angle=60,radius=2)
     -- (xyz polar cs:angle=90,radius=2);
 \end{tikzpicture}
\end{codeexample}

  The implicit version of this option is the same as the implicit
  version of |canvas polar|, only you do not provide a unit.

\begin{codeexample}[]
\tikz[x={(0cm,1cm)},y={(-1cm,0cm)}]
  \draw  (0,0) -- (30:1) -- (60:1) -- (90:1)
             -- (120:1) -- (150:1) -- (180:1);
\end{codeexample}
\end{coordinatesystem}

\begin{coordinatesystem}{xy polar}
  This is just an alias for |xyz polar|, which some people might
  prefer as there is no z-coordinate involved in the |xyz polar|
  coordinates.
\end{coordinatesystem}


\subsubsection{Barycentric Systems}
\label{section-barycentric-coordinates}

In the barycentric coordinate system a point is expressed as the
linear combination of multiple vectors. The idea is that you specify
vectors $v_1$, $v_2$, \dots, $v_n$ and numbers $\alpha_1$, $\alpha_2$,
\dots, $\alpha_n$. Then the barycentric coordinate specified by these
vectors and numbers is
\begin{align*}
  \frac{\alpha_1 v_1 + \alpha_2 v_2 + \cdots + \alpha_n v_n}{\alpha_1
    + \alpha_2 + \cdots + \alpha_n}
\end{align*}

The |barycentric cs| allows you to specify such coordinates easily.

\begin{coordinatesystem}{barycentric}
  For this coordinate system, the \meta{coordinate specification}
  should be a comma-separated list of expressions of the form
  \meta{node name}|=|\meta{number}. Note that (currently) the list
  should not contain any spaces before or after the \meta{node name}
  (unlike normal key-value pairs).

  The specified coordinate is now computed as follows: Each pair
  provides one vector and a number. The vector is the |center| anchor
  of the \meta{node name}. The number is the \meta{number}. Note that
  (currently) you cannot specify a different anchor, so that in order
  to use, say, the |north| anchor of a node you first have to create a
  new coordinate at this north anchor. (Using for instance
  \texttt{\string\coordinate (mynorth) at (mynode.north);}.)

\begin{codeexample}[]
\begin{tikzpicture}
  \coordinate (content)   at (90:3cm);
  \coordinate (structure) at (210:3cm);
  \coordinate (form)      at (-30:3cm);

  \node [above]       at (content)   {content oriented};
  \node [below left]  at (structure) {structure oriented};
  \node [below right] at (form)      {form oriented};

  \draw [thick,gray] (content.south) -- (structure.north east) -- (form.north west) -- cycle;

  \small
  \node at (barycentric cs:content=0.5,structure=0.1 ,form=1)    {PostScript};
  \node at (barycentric cs:content=1  ,structure=0   ,form=0.4)  {DVI};
  \node at (barycentric cs:content=0.5,structure=0.5 ,form=1)    {PDF};
  \node at (barycentric cs:content=0  ,structure=0.25,form=1)    {CSS};
  \node at (barycentric cs:content=0.5,structure=1   ,form=0)    {XML};
  \node at (barycentric cs:content=0.5,structure=1   ,form=0.4)  {HTML};
  \node at (barycentric cs:content=1  ,structure=0.2 ,form=0.8)  {\TeX};
  \node at (barycentric cs:content=1  ,structure=0.6 ,form=0.8)  {\LaTeX};
  \node at (barycentric cs:content=0.8,structure=0.8 ,form=1)    {Word};
  \node at (barycentric cs:content=1  ,structure=0.05,form=0.05) {ASCII};
\end{tikzpicture}
\end{codeexample}
\end{coordinatesystem}

\subsubsection{Node Coordinate System}
\label{section-node-coordinates}

In \pgfname\ and in \tikzname\ it is quite easy to define a node that you
wish to reference at a later point. Once you have defined a node,
there are different ways of referencing points of the node. To do so,
you use the following coordinate system:

\begin{coordinatesystem}{node}
  This coordinate system is used to reference a specific point inside
  or on the border of a previously defined node. It can be used in
  different ways, so let us go over them one by one.

  You can use three options to specify which coordinate you mean:
  \begin{key}{/tikz/cs/name=\meta{node name}}
    Specifies the node in which you which to specify a coordinate. The
    \meta{node name} is
    the name that was previously used to name the node using the
    |name=|\meta{node name} option or the special node name syntax.
  \end{key}
  \begin{key}{/tikz/anchor=\meta{anchor}}
    Specifies an anchor of the node. Here is an example:
\begin{codeexample}[]
\begin{tikzpicture}
  \node (shape)   at (0,2)  [draw] {|class Shape|};
  \node (rect)    at (-2,0) [draw] {|class Rectangle|};
  \node (circle)  at (2,0)  [draw] {|class Circle|};
  \node (ellipse) at (6,0)  [draw] {|class Ellipse|};

  \draw (node cs:name=circle,anchor=north) |- (0,1);
  \draw (node cs:name=ellipse,anchor=north) |- (0,1);
  \draw[-open triangle 90] (node cs:name=rect,anchor=north)
        |- (0,1) -| (node cs:name=shape,anchor=south);
\end{tikzpicture}
\end{codeexample}
  \end{key}
  \begin{key}{/tikz/cs/angle=\meta{degrees}}
    It is also possible to provide an angle \emph{instead} of an
    anchor. This coordinate refers to a point of the node's
    border where a ray shot from the center
    in the given angle hits the border. Here is an example:
\begin{codeexample}[]
\begin{tikzpicture}
  \node (start) [draw,shape=ellipse] {start};
  \foreach \angle in {-90, -80, ..., 90}
    \draw (node cs:name=start,angle=\angle)
      .. controls +(\angle:1cm) and +(-1,0) .. (2.5,0);
  \end{tikzpicture}
\end{codeexample}
  \end{key}

  It is possible to provide \emph{neither} the |anchor=| option nor
  the |angle=| option. In this case, \tikzname\ will calculate an
  appropriate border position for you. Here is an example:

\begin{codeexample}[]
\begin{tikzpicture}
  \path (0,0)  node(a) [ellipse,rotate=10,draw] {An ellipse}
        (3,-1) node(b) [circle,draw]            {A circle};
  \draw[thick] (node cs:name=a) -- (node cs:name=b);
\end{tikzpicture}
\end{codeexample}

  \tikzname\ will be reasonably clever at determining the border points that
  you ``mean,'' but, naturally, this may fail in some situations. If
  \tikzname\ fails to determine an appropriate border point, the center will
  be used instead.

  Automatic computation of anchors works only with the line-to operations
  |--|, the vertical/horizontal versions \verb!|-! and \verb!-|!, and
  with the curve-to operation |..|. For other path commands, such as
  |parabola| or |plot|, the center will be used. If this is not desired,
  you should give a named anchor or an angle anchor.

  Note that if you use an automatic coordinate for both the start and
  the end of a line-to, as in |--(node cs:name=b)--|, then \emph{two}
  border   coordinates are computed with a move-to between them. This
  is usually   exactly what you want.

  If you use relative coordinates together with automatic anchor
  coordinates, the relative coordinates are computed relative to
  the node's center, not relative to the border point. Here is an
  example:

\begin{codeexample}[]
\tikz \draw (0,0) node(x) [draw] {Text}
            rectangle (1,1)
            (node cs:name=x) -- +(1,1);
\end{codeexample}

Similarly, in the following examples both control points are $(1,1)$:

\begin{codeexample}[]
\tikz \draw (0,0) node(x) [draw] {X}
            (2,0) node(y) {Y}
            (node cs:name=x) .. controls +(1,1) and +(-1,1) ..
            (node cs:name=y);
\end{codeexample}

  The implicit way of specifying the node coordinate system is to
  simply use the name of the node in parentheses as in |(a)| or to
  specify a name together with an anchor or an angle separated by a
  dot as in |(a.north)| or |(a.10)|.

  Here is a more complete example:
\begin{codeexample}[]
\begin{tikzpicture}[fill=blue!20]
  \draw[help lines] (-1,-2) grid (6,3);
  \path (0,0)  node(a) [ellipse,rotate=10,draw,fill]    {An ellipse}
        (3,-1) node(b) [circle,draw,fill]               {A circle}
        (2,2)  node(c) [rectangle,rotate=20,draw,fill]  {A rectangle}
        (5,2)  node(d) [rectangle,rotate=-30,draw,fill] {Another rectangle};
  \draw[thick] (a.south) -- (b) -- (c) -- (d);
  \draw[thick,red,->] (a) |- +(1,3) -| (c) |- (b);
  \draw[thick,blue,<->] (b) .. controls +(right:2cm) and +(down:1cm) .. (d);
\end{tikzpicture}
\end{codeexample}
\end{coordinatesystem}


% Deprecated:

% \subsubsection{Intersection Coordinate Systems}

% Often you wish to specify a point that is on the
% intersection of two lines or shapes. For this, the following
% coordinate system is useful:

% \begin{coordinatesystem}{intersection}
%   First, you must specify two objects that should be
%   intersected. These ``objects'' can either be lines or the shapes of
%   nodes. There are two option to specify the first object:
%   \begin{key}{/tikz/cs/first line={\ttfamily\char`\{}|(|\meta{first
%           coordinate}|)--(|\meta{second coordinate}|)|{\ttfamily\char`\}}}
%     Specifies that the first object is a line that goes from
%     \meta{first coordinate} to meta{second coordinate}.
%   \end{key}
%   Note that you have to write |--| between the coordinate, but this
%   does not mean that anything is added to the path. This is simply a
%   special syntax.
%   \begin{key}{/tikz/cs/first node=\meta{node}}
%     Specifies that the first object is a previously defined node named
%     \meta{node}.
%   \end{key}

%   To specify the second object, you use one of the following keys:
%   \begin{key}{/tikz/cs/second line={\ttfamily\char`\{}|(|\meta{first
%           coordinate}|)--(|\meta{second coordinate}|)|{\ttfamily\char`\}}}
%     As above.
%   \end{key}
%   \begin{key}{/tikz/cs/second node=\meta{node}}
%     Specifies that the second object is a previously defined node
%     named \meta{node}.
%   \end{key}

%   Since it is possible that two objects have multiple intersections,
%   you may need to specify which solution you want:
%   \begin{key}{/tikz/cs/solution=\meta{number} (initially 1)}
%     Specifies which solution should be used. Numbering starts with 1.
%   \end{key}
%   The coordinate specified in this way is the \meta{number}th
%   intersection of the two objects.  If the objects do not intersect,
%   an error may occur.

% \begin{codeexample}[]
% \begin{tikzpicture}
%   \draw[help lines] (0,0) grid (3,2);
%   \draw (0,0) coordinate (A) -- (3,2) coordinate (B)
%         (1,2)                -- (3,0);

%   \fill[red] (intersection cs:
%     first line={(A)--(B)},
%     second line={(1,2)--(3,0)}) circle (2pt);
% \end{tikzpicture}
% \end{codeexample}

%   The implicit way of specifying this coordinate system is to write
%   \declare{|(intersection |\opt{\meta{number}}| of |\meta{first
%       object}%
%     | and |\meta{second object}|)|}. Here, \meta{first object} either
%   has the form \meta{$p_1$}|--|\meta{$p_2$} or it is just a node
%   name. Likewise for \meta{second object}. Note that there are \emph{no}
%   parentheses around the $p_i$. Thus, you would write
%   |(intersection of A--B and 1,2--3,0)|  for the intersection of the
%   line through the coordinates |A| and |B| and the line through the
%   points $(1,2)$ and $(3,0)$. You would write
%   |(intersection 2 of c_1 and c_2)| for the second
%   intersection of the node named |c_1| and the node named
%   |c_2|.

%   \tikzname\ needs an explicit algorithm for computing the
%   intersection of two shapes and such an algorithm is available only
%   for few shapes. Currently, the following intersection will be
%   computed correctly:
%   \begin{itemize}
%   \item a line and a line
%   \item a |circle| node and a line (in any order)
%   \item a |circle| and a |circle|
%   \end{itemize}
% \begin{codeexample}[]
% \begin{tikzpicture}[scale=.25]
%   \coordinate [label=-135:$a$] (a) at ($ (0,0)   + (rand,rand) $);
%   \coordinate [label=45:$b$]   (b) at ($ (3,2) + (rand,rand) $);

%   \coordinate [label=-135:$u$] (u) at (-1,1);
%   \coordinate [label=45:$v$]   (v) at (6,0);

%   \draw (a) -- (b)
%         (u) -- (v);

%   \node (c1) at (a) [draw,circle through=(b)] {};
%   \node (c2) at (b) [draw,circle through=(a)] {};

%   \coordinate [label=135:$c$] (c) at (intersection 2 of c1 and c2);
%   \coordinate [label=-45:$d$] (d) at (intersection of u--v and c2);
%   \coordinate [label=135:$e$] (e) at (intersection of u--v and a--b);

%   \foreach \p in {a,b,c,d,e,u,v}
%     \fill [opacity=.5] (\p) circle (8pt);
% \end{tikzpicture}
% \end{codeexample}
% \end{coordinatesystem}


\subsubsection{Tangent Coordinate Systems}

\begin{coordinatesystem}{tangent}
  This coordinate system, which is available only when the \tikzname\
  library |calc| is loaded, allows you to compute the point that lies
  tangent to a shape. In detail, consider a \meta{node} and a
  \meta{point}. Now, draw a straight line from the \meta{point} so
  that it ``touches'' the \meta{node} (more formally, so that it is
  \emph{tangent} to this \meta{node}). The point where the line
  touches the shape is the point referred to by the |tangent|
  coordinate system.

  The following options may be given:
  \begin{key}{/tikz/cs/node=\meta{node}}
    This key specifies the node on whose border the tangent should
    lie.
  \end{key}
  \begin{key}{/tikz/cs/point=\meta{point}}
    This key specifies the point through which the tangent should go.
  \end{key}
  \begin{key}{/tikz/cs/solution=\meta{number}}
    Specifies which solution should be used if there are more than one.
  \end{key}

  A special algorithm is needed in order to compute the tangent for a
  given shape. Currently, tangents can be computed for nodes whose
  shape is one of the following:
  \begin{itemize}
  \item |coordinate|
  \item |circle|
  \end{itemize}

\begin{codeexample}[]
\begin{tikzpicture}
  \draw[help lines] (0,0) grid (3,2);

  \coordinate (a) at (3,2);

  \node [circle,draw] (c) at (1,1) [minimum size=40pt] {$c$};

  \draw[red] (a)  -- (tangent cs:node=c,point={(a)},solution=1) --
       (c.center) -- (tangent cs:node=c,point={(a)},solution=2) -- cycle;
\end{tikzpicture}
\end{codeexample}

  There is no implicit syntax for this coordinate system.
\end{coordinatesystem}



\subsubsection{Defining New Coordinate Systems}

While the set of coordinate systems that \tikzname\ can parse via
their special syntax is fixed, it is possible and quite easy to define
new explicitly named coordinate systems. For this, the following
commands are used:

\begin{command}{\tikzdeclarecoordinatesystem\marg{name}\marg{code}}
  This command declares a new coordinate system named \meta{name} that
  can later on be used by writing
  |(|\meta{name}| cs:|\meta{arguments}|)|. When \tikzname\ encounters a coordinate
  specified in this way, the \meta{arguments} are passed to
  \meta{code} as argument |#1|.

  It is now the job of \meta{code} to make sense of the
  \meta{arguments}. At the end of \meta{code}, the two \TeX\ dimensions
  |\pgf@x| and |\pgf@y| should be have the $x$- and $y$-canvas
  coordinate of the coordinate.

  It is not necessary, but customary, to parse \meta{arguments} using
  the key-value syntax. However, you can also parse it in any way you
  like.

  In the following example, a coordinate system |cylindrical| is
  defined.
\begin{codeexample}[]
\makeatletter
\define@key{cylindricalkeys}{angle}{\def\myangle{#1}}
\define@key{cylindricalkeys}{radius}{\def\myradius{#1}}
\define@key{cylindricalkeys}{z}{\def\myz{#1}}
\tikzdeclarecoordinatesystem{cylindrical}%
{%
  \setkeys{cylindricalkeys}{#1}%
  \pgfpointadd{\pgfpointxyz{0}{0}{\myz}}{\pgfpointpolarxy{\myangle}{\myradius}}
}
\begin{tikzpicture}[z=0.2pt]
  \draw [->] (0,0,0) -- (0,0,350);
  \foreach \num in {0,10,...,350}
    \fill (cylindrical cs:angle=\num,radius=1,z=\num) circle (1pt);
\end{tikzpicture}
\end{codeexample}
\end{command}

\begin{command}{\tikzaliascoordinatesystem\marg{new name}\marg{old name}}
  Creates an alias of \meta{old name}.
\end{command}



\subsection{Coordinates at Intersections}
\label{section-intersection-coordinates}

You will wish to compute the intersection of two paths. For the
special and frequent case of two perpendicular lines, a special
coordinate system called |perpendicular| is available. For more
general cases, the |intersection| library can be used.


\subsubsection{Intersections of Perpendicular Lines}

A frequent special case of path intersections is the intersection of a
vertical line going through a point $p$ and a horizontal line going
through some other point $q$. For this situation there is a useful
coordinate system.

\begin{coordinatesystem}{perpendicular}
  You can specify the two lines using the following keys:

  \begin{key}{/tikz/cs/horizontal line through={\ttfamily\char`\{}|(|\meta{coordinate}|)|{\ttfamily\char`\}}}
    Specifies that one line is a horizontal line that goes through the
    given coordinate.
  \end{key}
  \begin{key}{/tikz/cs/vertical line through={\ttfamily\char`\{}|(|\meta{coordinate}|)|{\ttfamily\char`\}}}
    Specifies that the other line is vertical and goes through the
    given coordinate.
  \end{key}

  However, in almost all cases you should, instead, use the implicit
  syntax. Here, you write \declare{|(|\meta{p}\verb! |- !\meta{q}|)|} or
  \declare{|(|\meta{q}\verb! -| !\meta{p}|)|}.

  For example, \verb!(2,1 |- 3,4)! and  \verb!(3,4 -| 2,1)! both yield
  the same as \verb!(2,4)! (provided the $xy$-coordinate system has not
  been modified).

  The most useful application of the syntax is to draw a line up to some
  point on a vertical or horizontal line. Here is an example:

\begin{codeexample}[]
\begin{tikzpicture}
  \path (30:1cm) node(p1) {$p_1$}   (75:1cm) node(p2) {$p_2$};

  \draw (-0.2,0) -- (1.2,0) node(xline)[right] {$q_1$};
  \draw (2,-0.2) -- (2,1.2) node(yline)[above] {$q_2$};

  \draw[->] (p1) -- (p1 |- xline);
  \draw[->] (p2) -- (p2 |- xline);
  \draw[->] (p1) -- (p1 -| yline);
  \draw[->] (p2) -- (p2 -| yline);
\end{tikzpicture}
\end{codeexample}
\end{coordinatesystem}


\subsubsection{Intersections of Arbitrary Paths}

\begin{tikzlibrary}{intersections}
  This library enables the calculation of intersections of
  two arbitrary paths. However, due to the low accuracy of
  \TeX, the paths should not be ``too complicated''.
  In particular, you should not try to intersect paths consisting of
  lots of very small segments such as plots or decorated paths.
\end{tikzlibrary}

To find the intersections of two paths in \tikzname, they must be
``named''. A ``named path'' is, quite simply, a path that has been
named using the following key:

\begin{keylist}{%
	/tikz/name path=\meta{name},
	/tikz/name path global=\meta{name}}
  The effect of this key is that, after the path has been constructed,
  just before it is used, it is associated with \meta{name}. For |name path|,
  this association survives beyond the final semi-colon of the path
  but not the end of the surrounding scope. For |name path global|, the association
  will survive beyond any scope as well. Handle with care.

  Any paths created by nodes on the (main) path are ignored, unless
  this key is explicitly used. If the same \meta{name} is used for the
  main path and the node path(s), then the paths will be added
  together and then associated with \meta{name}.
\end{keylist}

To find the intersection of named paths, the following key is used:

\begin{key}{/tikz/name intersections=\marg{options}}
  This key changes the key path to |/tikz/intersection| and processes
  \meta{options}. These options determine, among other things,
  which paths to use for the intersection. Having processed the
  options, any intersections are then found. A coordinate is created
  at each intersection, which by default, will be named
  |intersection-1|, |intersection-2|, and so on.
  Optionally, the prefix |intersection| can be changed, and the
  total number of intersections stored in a \TeX-macro.

\begin{codeexample}[]
\begin{tikzpicture}[every node/.style={opacity=1, black, above left}]
  \draw [help lines] grid (3,2);
  \draw [name path=ellipse] (2,0.5) ellipse (0.75cm and 1cm);
  \draw [name path=rectangle, rotate=10] (0.5,0.5) rectangle +(2,1);
  \fill [red, opacity=0.5, name intersections={of=ellipse and rectangle}]
    (intersection-1) circle (2pt) node {1}
    (intersection-2) circle (2pt) node {2};
\end{tikzpicture}
\end{codeexample}

The following keys can be used in \meta{options}:

\begin{key}{/tikz/intersection/of=\meta{name path 1}| and |\meta{name path 2}}
  This key is used to specify the names of the paths to use for
  the intersection.
\end{key}

\begin{key}{/tikz/intersection/name=\meta{prefix} (initially intersection)}
  This key specifies the prefix name for the coordinate nodes placed
  at each intersection.
\end{key}

\begin{key}{/tikz/intersection/total=\meta{macro}}
  This key means that the total number of intersections found
  will be stored in \meta{macro}.
\end{key}

\begin{codeexample}[]
\begin{tikzpicture}
  \clip (-2,-2) rectangle (2,2);
  \draw [name path=curve 1] (-2,-1) .. controls (8,-1) and (-8,1) .. (2,1);
  \draw [name path=curve 2] (-1,-2) .. controls (-1,8) and (1,-8) .. (1,2);

  \fill [name intersections={of=curve 1 and curve 2, name=i, total=\t}]
        [red, opacity=0.5, every node/.style={above left, black, opacity=1}]
        \foreach \s in {1,...,\t}{(i-\s) circle (2pt) node {\footnotesize\s}};
\end{tikzpicture}
\end{codeexample}


  \begin{key}{/tikz/intersection/by=\meta{comma-separated list}}
    This key allows you to specify a list of names for the intersection
    coordinates. The intersection coordinates will still be named
    \meta{prefix}|-|\meta{number}, but additionally the first
    coordinate will also be named by the first element of the
    \meta{comma-separated list}. What happens is that the
    \meta{comma-separated list} is passed to the |\foreach| statement
    and for \meta{list member} a coordinate is created at the
    already-named intersection.
\begin{codeexample}[]
\begin{tikzpicture}
  \clip (-2,-2) rectangle (2,2);
  \draw [name path=curve 1] (-2,-1) .. controls (8,-1) and (-8,1) .. (2,1);
  \draw [name path=curve 2] (-1,-2) .. controls (-1,8) and (1,-8) .. (1,2);

  \fill [name intersections={of=curve 1 and curve 2, by={a,b}}]
        (a) circle (2pt)
        (b) circle (2pt);
\end{tikzpicture}
\end{codeexample}

    You can also use the |...| notation of the |\foreach| statement
    inside the \meta{comma-separated list}.

    In case an element of the \meta{comma-separated list} starts with
    options in square brackets, these options are used when the
    coordinate is created. A coordinate name can still, but need not,
    follow the  options. This
    makes it easy to add labels to intersections:
\begin{codeexample}[]
\begin{tikzpicture}
  \clip (-2,-2) rectangle (2,2);
  \draw [name path=curve 1] (-2,-1) .. controls (8,-1) and (-8,1) .. (2,1);
  \draw [name path=curve 2] (-1,-2) .. controls (-1,8) and (1,-8) .. (1,2);

  \fill [name intersections={
          of=curve 1 and curve 2,
          by={[label=center:a],[label=center:...],[label=center:i]}}];
\end{tikzpicture}
\end{codeexample}
  \end{key}

  \begin{key}{/tikz/intersection/sort by=\meta{path name}}
By default, the intersections are simply returned in the order that
the intersection algorithm finds them. Unfortunately, this is not
necessarily a ``helpful'' ordering. This key can be used to sort
the intersections along the path specified by \meta{path name},
which should be one of the paths mentioned in the
|/tikz/intersection/of| key.

\begin{codeexample}[]
\begin{tikzpicture}
\clip (-0.5,-0.75) rectangle (3.25,2.25);
\foreach \pathname/\shift in {line/0cm, curve/2cm}{
  \tikzset{xshift=\shift}
  \draw [->, name path=curve] (1,1.5) .. controls (-1,1) and (2,0.5) .. (0,0);
  \draw [->, name path=line]  (0,-.5) -- (1,2) ;
  \fill [name intersections={of=line and curve,sort by=\pathname, name=i}]
    [red, opacity=0.5, every node/.style={left=.25cm, black, opacity=1}]
    \foreach \s in {1,2,3}{(i-\s) circle (2pt) node {\footnotesize\s}};
}
\end{tikzpicture}
\end{codeexample}

  \end{key}
\end{key}




\subsection{Relative and Incremental Coordinates}


\subsubsection{Specifying Relative Coordinates}

You can prefix coordinates by |++| to make them ``relative.'' A
coordinate such as |++(1cm,0pt)| means ``1cm to the right of the
previous position.'' Relative coordinates are often useful in
``local'' contexts:

\begin{codeexample}[]
\begin{tikzpicture}
  \draw (0,0)     -- ++(1,0) -- ++(0,1) -- ++(-1,0) -- cycle;
  \draw (2,0)     -- ++(1,0) -- ++(0,1) -- ++(-1,0) -- cycle;
  \draw (1.5,1.5) -- ++(1,0) -- ++(0,1) -- ++(-1,0) -- cycle;
\end{tikzpicture}
\end{codeexample}

Instead of |++| you can also use a single |+|. This also specifies a
relative coordinate, but it does not ``update'' the current point for
subsequent usages of relative coordinates. Thus, you can use this
notation to specify numerous points, all relative to the same
``initial'' point:

\begin{codeexample}[]
\begin{tikzpicture}
  \draw (0,0)     -- +(1,0) -- +(1,1) -- +(0,1) -- cycle;
  \draw (2,0)     -- +(1,0) -- +(1,1) -- +(0,1) -- cycle;
  \draw (1.5,1.5) -- +(1,0) -- +(1,1) -- +(0,1) -- cycle;
\end{tikzpicture}
\end{codeexample}

There is a special situation, where relative coordinates are
interpreted differently. If you use a relative coordinate as a control
point of a B\'ezier curve, the following rule applies: First, a relative
first control point is taken relative to the beginning of the
curve. Second, a relative second control point is taken relative to
the end of the curve. Third, a relative end point of a curve is taken
relative to the start of the curve.

This special behavior makes it easy to specify that a curve should
``leave or arrive from a certain direction'' at the start or end. In
the following example, the curve ``leaves'' at $30^\circ$ and
``arrives'' at $60^\circ$:

\begin{codeexample}[]
\begin{tikzpicture}
  \draw (1,0) .. controls +(30:1cm) and +(60:1cm) .. (3,-1);
  \draw[gray,->] (1,0) -- +(30:1cm);
  \draw[gray,<-] (3,-1) -- +(60:1cm);
\end{tikzpicture}
\end{codeexample}


\subsubsection{Relative Coordinates and Scopes}
\label{section-scopes-relative}
An interesting question is, how do relative coordinates behave in the
presence of scopes? That is, suppose we use curly braces in a path to
make part of it ``local,'' how does that affect the current position?
On the one hand, the current position certainly changes since the
scope only affects options, not the path itself. On the other hand, it
may be useful to ``temporarily escape'' from the updating of the
current point.

Since both interpretations of how the current point and scopes should
``interact'' are useful, there is a (local!) option that allows you to
decide which you need.

\begin{key}{/tikz/current point is local=\opt{\meta{boolean}} (initially
    false)}
  Normally, the scope path operation has no effect on the current
  point. That is, curly braces on a path have no effect on the current
  position:
\begin{codeexample}[]
\begin{tikzpicture}
  \draw      (0,0) -- ++(1,0)   -- ++(0,1)   -- ++(-1,0);
  \draw[red] (2,0) -- ++(1,0) { -- ++(0,1) } -- ++(-1,0);
\end{tikzpicture}
\end{codeexample}
  If you set this key to |true|, this behaviour changes. In this case,
  at the end of a group created on a path, the last current position
  reverts to whatever value it had at the beginning of the scope. More
  precisely, when \tikzname\ encounters |}| on a path, it checks
  whether at this particular moment the key is set to |true|. If so,
  the current position reverts to the value is had when the matching
  |{| was read.
\begin{codeexample}[]
\begin{tikzpicture}
  \draw      (0,0) -- ++(1,0)   -- ++(0,1)   -- ++(-1,0);
  \draw[red] (2,0) -- ++(1,0)
     { [current point is local] -- ++(0,1) } -- ++(-1,0);
\end{tikzpicture}
\end{codeexample}
  In the above example, we could also have given the option outside
  the scope, for instance as a parameter to the whole scope.
\end{key}


\subsection{Coordinate Calculations}
\label{tikz-lib-calc}

\begin{tikzlibrary}{calc}
  You need to load this library in order to use the coordinate
  calculation functions described in the present section.
\end{tikzlibrary}


It is possible to do some basic calculations that involve
coordinates. In essence, you can add and subtract coordinates, scale
them, compute midpoints, and do projections. For instance,
|($(a) + 1/3*(1cm,0)$)| is the coordinate that is $1/3 \text{cm}$ to the right
of the point |a|:
\begin{codeexample}[]
\begin{tikzpicture}
  \draw [help lines] (0,0) grid (3,2);

  \node (a) at (1,1) {A};
  \fill [red] ($(a) + 1/3*(1cm,0)$) circle (2pt);
\end{tikzpicture}
\end{codeexample}



\subsubsection{The General Syntax}

The general syntax is the following:

\begin{quote}
  \declare{|(|\opt{|[|\meta{options}|]|}|$|\meta{coordinate computation}|$)|}.
\end{quote}

As you can see, the syntax uses the \TeX\ math symbol |$| to %$
indicate that a ``mathematical computation'' is involved. However, the |$| %$
has no other effect, in particular, no mathematical text is typeset.

The \meta{coordinate computation} has the following structure:
\begin{enumerate}
\item
  It starts with
  \begin{quote}
    \opt{\meta{factor}|*|}\meta{coordinate}\opt{\meta{modifiers}}
  \end{quote}
\item
  This is optionally followed by |+| or |-| and then another
  \begin{quote}
    \opt{\meta{factor}|*|}\meta{coordinate}\opt{\meta{modifiers}}
  \end{quote}
\item
  This is once more followed by |+| or |-| and another of the above
  modified coordinate; and so on.
\end{enumerate}

In the following, the syntax of factors and of the different modifiers
is explained in detail.


\subsubsection{The Syntax of Factors}

The \meta{factor}s are optional and detected
by checking whether the \meta{coordinate computation} starts with a
|(|. Also, after each $\pm$ a \meta{factor} is present if, and only
if, the |+| or |-| sign is not directly followed by~|(|.

If a \meta{factor} is present, it is evaluated using the
|\pgfmathparse| macro. This means that you can use pretty complicated
computations inside a factor. A \meta{factor} may even contain opening
parentheses, which creates a complication: How does \tikzname\ know
where a \meta{factor} ends and where a coordinate starts? For
instance, if the beginning of a \meta{coordinate computation} is
|2*(3+4|\dots, it is not clear whether |3+4| is part of a
\meta{coordinate} or part of a \meta{factor}. Because of this, the
following rule is used: Once it has been determined, that a
\meta{factor} is present, in principle, the \meta{factor} contains
everything up to the next occurrence of |*(|. Note that there is no
space between the asterisk and the parenthesis.

It is permissible to put the \meta{factor} in curly braces. This can
be used whenever it is unclear where the \meta{factor} would end.

Here are some examples of coordinate specifications that consist of
exactly one \meta{factor} and one \meta{coordinate}:
\begin{codeexample}[]
\begin{tikzpicture}
  \draw [help lines] (0,0) grid (3,2);

  \fill [red] ($2*(1,1)$) circle (2pt);
  \fill [green] (${1+1}*(1,.5)$) circle (2pt);
  \fill [blue] ($cos(0)*sin(90)*(1,1)$) circle (2pt);
  \fill [black] (${3*(4-3)}*(1,0.5)$) circle (2pt);
\end{tikzpicture}
\end{codeexample}



\subsubsection{The Syntax of Partway Modifiers}

A \meta{coordinate} can be followed by different \meta{modifiers}. The
first kind of modifier is the \emph{partway modifier}. The syntax
(which is loosely inspired by Uwe Kern's |xcolor| package) is the
following:
\begin{quote}
  \meta{coordinate}\declare{|!|\meta{number}|!|\opt{\meta{angle}|:|}\meta{second coordinate}}
\end{quote}
One could write for instance
\begin{codeexample}[code only]
(1,2)!.75!(3,4)
\end{codeexample}
The meaning of this is: ``Use the coordinate that is three quarters on
the way from |(1,2)| to |(3,4)|.'' In general, \meta{coordinate
  x}|!|\meta{number}|!|\meta{coordinate y} yields the coordinate
$(1-\meta{number})\meta{coordinate x} + \meta{number} \meta{coordinate
  y}$. Note that this is a bit different from the way the
\meta{number} is interpreted in the |xcolor| package: First, you use a
factor between $0$ and $1$, not a percentage, and, second, as the
\meta{number} approaches $1$, we approach the second coordinate, not
the first. It is permissible to use a \meta{number} that is smaller
than $0$ or larger than $1$. The \meta{number} is evaluated using the
|\pgfmathparse| command and, thus, it can involve complicated
computations.

\begin{codeexample}[]
\begin{tikzpicture}
  \draw [help lines] (0,0) grid (3,2);

  \draw (1,0) -- (3,2);

  \foreach \i in {0,0.2,0.5,0.9,1}
    \node at ($(1,0)!\i!(3,2)$) {\i};
\end{tikzpicture}
\end{codeexample}

The \meta{second coordinate} may be prefixed by an \meta{angle},
separated with a colon, as in |(1,1)!.5!60:(2,2)|. The general meaning
of \meta{a}|!|\meta{factor}|!|\meta{angle}|:|\meta{b} is ``First,
consider the line from \meta{a} to \meta{b}. Then rotate this line by
\meta{angle} \emph{around the point \meta{a}}. Then the two endpoints
of this line will be \meta{a} and some point \meta{c}. Use this point
\meta{c} for the subsequent computation, namely the partway
computation.''

Here are two examples:
\begin{codeexample}[]
\begin{tikzpicture}
  \draw [help lines] (0,0) grid (3,3);

  \coordinate (a) at (1,0);
  \coordinate (b) at (3,2);

  \draw[->] (a) -- (b);

  \coordinate (c) at ($ (a)!1! 10:(b) $);

  \draw[->,red] (a) -- (c);

  \fill ($ (a)!.5! 10:(b) $) circle (2pt);
\end{tikzpicture}
\end{codeexample}


\begin{codeexample}[]
\begin{tikzpicture}
  \draw [help lines] (0,0) grid (4,4);

  \foreach \i in {0,0.1,...,2}
    \fill ($(2,2) !\i! \i*180:(3,2)$) circle (2pt);
\end{tikzpicture}
\end{codeexample}


You can repeatedly apply modifiers. That is, after any modifier
you can add another (possibly different) modifier.

\begin{codeexample}[]
\begin{tikzpicture}
  \draw [help lines] (0,0) grid (3,2);

  \draw (0,0) -- (3,2);
  \draw[red] ($(0,0)!.3!(3,2)$) -- (3,0);
  \fill[red] ($(0,0)!.3!(3,2)!.7!(3,0)$) circle (2pt);
\end{tikzpicture}
\end{codeexample}


\subsubsection{The Syntax of Distance Modifiers}

A \emph{distance modifier} has nearly the same syntax as a partway
modifier, only you use a \meta{dimension} (something like |1cm|)
instead of a \meta{factor} (something like |0.5|):
\begin{quote}
  \meta{coordinate}\declare{|!|\meta{dimension}|!|\opt{\meta{angle}|:|}\meta{second coordinate}}
\end{quote}

When you write \meta{a}|!|\meta{dimension}|!|\meta{b}, this means the
following: Use the point that is distanced \meta{dimension} from
\meta{a} on the straight line from \meta{a} to \meta{b}. Here is an example:
\begin{codeexample}[]
\begin{tikzpicture}
  \draw [help lines] (0,0) grid (3,2);

  \draw (1,0) -- (3,2);

  \foreach \i in {0cm,1cm,15mm}
    \node at ($(1,0)!\i!(3,2)$) {\i};
\end{tikzpicture}
\end{codeexample}

As before, if you use a \meta{angle}, the \meta{second coordinate} is
rotated by this much around the \meta{coordinate} before it is used.

The combination of an \meta{angle} of |90| degrees with a distance can
be used to ``offset'' a point relative to a line. Suppose, for
instance, that you have computed a point |(c)| that lies somewhere on
a line from |(a)| to~|(b)| and you now wish to offset this point by
|1cm| so that the distance from this offset point to the line is
|1cm|. This can be achieved as follows:
\begin{codeexample}[]
\begin{tikzpicture}
  \draw [help lines] (0,0) grid (3,2);

  \coordinate (a) at (1,0);
  \coordinate (b) at (3,1);

  \draw (a) -- (b);

  \coordinate (c) at ($ (a)!.25!(b) $);
  \coordinate (d) at ($ (c)!1cm!90:(b) $);

  \draw [<->] (c) -- (d) node [sloped,midway,above] {1cm};
\end{tikzpicture}
\end{codeexample}



\subsubsection{The Syntax of Projection Modifiers}

The projection modifier is also similar to the above modifiers: It also
gives a point on a line from the \meta{coordinate} to the \meta{second
  coordinate}. However, the \meta{number} or \meta{dimension} is replaced by a
\meta{projection coordinate}:
\begin{quote}
  \meta{coordinate}\declare{|!|\meta{projection coordinate}|!|\opt{\meta{angle}|:|}\meta{second coordinate}}
\end{quote}

Here is an example:
\begin{codeexample}[code only]
(1,2)!(0,5)!(3,4)
\end{codeexample}

The effect is the following: We project the \meta{projection
  coordinate} orthogonally onto the line from \meta{coordinate} to
\meta{second coordinate}. This makes it easy to compute projected
points:
\begin{codeexample}[]
\begin{tikzpicture}
  \draw [help lines] (0,0) grid (3,2);

  \coordinate (a) at (0,1);
  \coordinate (b) at (3,2);
  \coordinate (c) at (2.5,0);

  \draw (a) -- (b) -- (c) -- cycle;

  \draw[red]    (a) -- ($(b)!(a)!(c)$);
  \draw[orange] (b) -- ($(a)!(b)!(c)$);
  \draw[blue]   (c) -- ($(a)!(c)!(b)$);
\end{tikzpicture}
\end{codeexample}
