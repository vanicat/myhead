% Copyright 2010 by Till Tantau
%
% This file may be distributed and/or modified
%
% 1. under the LaTeX Project Public License and/or
% 2. under the GNU Free Documentation License.
%
% See the file doc/generic/pgf/licenses/LICENSE for more details.


\section{Axes}
\label{section-dv-axes}

\subsection{Overview}

To be written...


\subsection{Concepts}

\subsubsection{Axes}

\subsubsection{Mayor, Minor, and Subminor Ticks}

\subsubsection{Tick Placement Strategies}

Consider the following problem: The data visualization engine determines that in a
plot the $x$-values vary between $17.4$ and $34.5$. In this case, we 
certainly do not want, say, ten ticks at exactly ten evenly spaced
positions starting with $17.4$ and ending with $34.5$, because this
would yield ticks at positions like $32.6$. Ticks should be placed at
``nice'' positions like $20$, $25$, and $30$.

Determining which positions are ``nice'' is somewhat difficult. In the
above example, the positions $20$, $25$, and $30$ are certainly nice,
but only three ticks may be a bit few of them. Better might be the
tick positions $17.5$, $20$, $22.5$, through to $32.5$. However, users
might prefer even numbers over fractions like $2.5$ as the stepping.

A \emph{tick placement strategy} is a method of automatically deciding
which positions are \emph{good} for placing ticks. The data
visualization engine comes with a number of predefined strategies, see
Section~\ref{section-dv-tick-placement-strategies}, but you can also
define new ones yourself.
Here is an example of the different stepping chosen when one varies
the tick placement strategy:

\begin{codeexample}[]
\begin{tikzpicture}
  \datavisualization [scientific axes, visualize as smooth line] 
    data [format=function] {
      var x : interval [1:11];
      func y = \value x*\value x;
    };
\end{tikzpicture}
\qquad
\begin{tikzpicture}
  \datavisualization [scientific axes, visualize as smooth line,
    y axis={exponential steps},
    x axis={ticks={quarter about strategy}},
  ] 
    data [format=function] {
      var x : interval [1:11];
      func y = \value x*\value x;
    };
\end{tikzpicture}
\end{codeexample}



\subsubsection{Grids}

\subsection{Usage}


\subsection{Reference: Standard Axis Systems}

In this section the axis system commonly used in data visualizations
are described. 



\subsubsection{Scientific Axis Systems}

\begin{key}{/tikz/data visualization/scientific axes}
  This key installs a two-dimensional coordinate system based on the
  attributes |/data point/x| and |/data point/y|.
      
\begin{codeexample}[]
\begin{tikzpicture}
  \datavisualization [scientific axes, visualize as smooth line] 
    data [format=function] {
      var x : interval [0:100] samples 100;
      func y = sqrt(\value x);
    };
\end{tikzpicture}
\end{codeexample}

  This axis system is usually a good choice to depict ``arbitrary two
  dimensional data.'' Because the axes are automatically scaled, you
  do not need to worry about how large or small the values will
  be. The name |scientific axes| is intended to indicate that this
  axis system is often used in scientific publications.

  Note, however, that this axis system will always distort the
  relative magnitudes of the units on the two axis. If you wish the
  units on both axes to be equal, consider directly specifying the
  unit length ``by hand'':

\begin{codeexample}[]
\begin{tikzpicture}
  \datavisualization [visualize as smooth line,
                      scientific axes,
                      all axes={unit length=1cm per 10 units, ticks={few}}] 
    data [format=function] {
      var x : interval [0:100] samples 100;
      func y = sqrt(\value x);
    };
\end{tikzpicture}
\end{codeexample}

  The |scientific axes| have the following properties:
  \begin{itemize}
  \item The |x|-values are surveyed and the $x$-axis is then scaled
    and shifted so  that it has the length specified by the following key.
    \begin{key}{/tikz/data visualization/scientific
        axes/width=\meta{dimension} (initially 5cm)} 
    \end{key}
    The minimum value is at the left end of the axis and at the canvas
    origin. The maximum value is at the right end of the axis.
  \item The |y|-values are surveyd and the $y$-axis is then scaled so
    that is has the length specified by the following key.
    \begin{key}{/tikz/data visualization/scientific
        axes/height=\meta{dimension}} 
      By default, the |height| is the golden ratio times the |width|.
    \end{key}
    The minimum value is at the bottom of the axis and at the canvas
    origin. The maximum value is at the top of the axis.
  \item Lines (forming a frame) are depicted at the minimum and
    maximum values of the axes in 50\% black.
  \item Ticks are drawn `` on the outside'' of the frame so that they
    interfere as little as possible with the data.
  \item Tick labels and axis labels (if present) are drawn left and
    below. 
  \end{itemize}
\end{key}

\begin{key}{/tikz/data visualization/scientific inner axes}
  This axis system works like |scientic axes|, only the ticks are on
  the ``inside'' of the frame. 
      
\begin{codeexample}[]
\begin{tikzpicture}
  \datavisualization [scientific inner axes, visualize as smooth line] 
    data [format=function] {
      var x : interval [-12:12];
      func y = \value x*\value x*\value x;
    };
\end{tikzpicture}
\end{codeexample}

  This axis system is also common in publications, but the ticks tend
  to interfere with marks if they are near to the border as can be
  seen in the following example:
\begin{codeexample}[]
\begin{tikzpicture}
  \datavisualization [scientific inner axes, scientific axes/width=3.2cm,
                      visualize as scatter/.list={a,b}] 
    data [a] {
      x, y
      0, 0
      1, 1
      0.5, 0.5
      2, 1
    }
    data [b] {
      x, y
      0.05, 0
      1.5, 1
      0.5, 0.75
      2, 0.5
    };
\end{tikzpicture}
\end{codeexample}

\end{key}

\begin{key}{/tikz/data visualization/scientific clean axes}
  This axis system is another version of |scientic axes|. However, the
  axes and the ticks are completely removed from the actual data,
  making this axis system especially useful for scatter plots, but
  also for most other scientific plots.
      
\begin{codeexample}[]
\begin{tikzpicture}
  \datavisualization [scientific clean axes, visualize as smooth line] 
    data [format=function] {
      var x : interval [-12:12];
      func y = \value x*\value x*\value x;
    };
\end{tikzpicture}
\end{codeexample}

  The distance of the axes from the actual plot is given by the
  padding of the axes.
\end{key}


For all scientific axis systems, different label placement strategies
can be specified. They are discussed in the following.


\begin{key}{/tikz/data visualization/scientific axes standard labels}
  As the name suggests, this is the standard placement strategy. The
  label of the $x$-axis is placed below the center of the $x$-axis,
  the label of the $y$-axis is rotated by $90^\circ$ and placed left
  of the center of the $y$-axis.
\begin{codeexample}[]
\begin{tikzpicture}
  \datavisualization [scientific clean axes,
                      visualize as smooth line,
                      scientific axes standard labels,
                      x axis={label=degree $d$, ticks={tick unit=${}^\circ$}},
                      y axis={label=$\sin d$}]
    data [format=function] {
      var x : interval [-10:10] samples 10;
      func y = sin(\value x);
    };
\end{tikzpicture}
\end{codeexample}
\end{key}

\begin{key}{/tikz/data visualization/scientific axes upright labels}
  Works like |scientific axes standard labels|, only the label of the
  $y$-axis is not rotated.
\begin{codeexample}[]
\begin{tikzpicture}
  \datavisualization [scientific clean axes,
                      visualize as smooth line,
                      scientific axes upright labels,
                      x axis={label=degree $d$, ticks={tick unit=${}^\circ$}},
                      y axis={label=$\cos d$,
                              ticks={style={/pgf/number format/.cd,precision=4,fixed zerofill}}}]
    data [format=function] {
      var x : interval [-10:10] samples 10;
      func y = cos(\value x);
    };
\end{tikzpicture}
\end{codeexample}
\end{key}


\begin{key}{/tikz/data visualization/scientific axes end labels}
  Places the labels at the end of the $x$- and the $y$-axis, similar
  to the axis labels of a school book axis system.
\begin{codeexample}[]
\begin{tikzpicture}
  \datavisualization [scientific clean axes,
                      visualize as smooth line,
                      scientific axes end labels,
                      x axis={label=degree $d$, ticks={tick unit=${}^\circ$}},
                      y axis={label=$\tan d$}]
    data [format=function] {
      var x : interval [-10:10] samples 10;
      func y = tan(\value x);
    };
\end{tikzpicture}
\end{codeexample}
\end{key}





\subsubsection{School Book Axis Systems}

\begin{key}{/tikz/data visualization/school book axes}
  This axis system is intended to ``look like'' the coordinate systems
  often used in school books: The axes are drawn in such a way that
  they intersect to origin. Furthermore, no automatic
  scaling is done to ensure that the lenghts of units are the same in
  all directions.

  This axis system must be used with care -- it is nearly always
  necessary to specify the desired unit length by hand using the
  option |unit length|. If the magnitudes of the units on the two axes
  differ, different unit lengths typically need to be specified for
  the different axes.

  Finally, if the data is ``far removed'' from the origin, this
  axis system will also ``look bad.''

\begin{codeexample}[]
\begin{tikzpicture}
  \datavisualization [school book axes, visualize as smooth line] 
    data [format=function] {
      var x : interval [-1.3:1.3];
      func y = \value x*\value x*\value x;
    };
\end{tikzpicture}
\end{codeexample}

  The stepping of the ticks is one unit by default. Using keys like
  |ticks=some| may help to give better steppings.
\end{key}


\begin{key}{/tikz/data visualization/school book axes standard labels}
  This key makes the label of the $x$-axis appear at the right end of
  this axis and it makes the label of the $y$-axis appear at the top
  of the $y$-axis.

  Currently, this is the only supported placement strategy for the
  school book axis system.
\begin{codeexample}[]
\begin{tikzpicture}
  \datavisualization [school book axes,
                      visualize as smooth line,
                      school book axes standard labels,
                      clean ticks,
                      x axis={label=$x$},
                      y axis={label=$f(x)$}]
    data [format=function] {
      var x : interval [-1:1];
      func y = \value x*\value x + 1;
    };
\end{tikzpicture}
\end{codeexample}
\end{key}





\subsubsection{Advanced: Underlying Cartesian Axis Systems}

The axis systems described in the following are typically not used
directly by the user. The systems setup \emph{directions} for several
axes in some sensible way, but they do not actually draw anything on
these axes. For instance, the |xy Cartesian| creates two axes called
|x axis| and |y axis| and makes the $x$-axis point right and the
$y$-axis point up. In contrast, an axis system like |scientific axes|
uses the axis system |xy Cartesian| internally and then proceeds to
setup a lot of keys so that the axis lines are drawn,
ticks and grid lines are drawn, and labels are placed at the correct
positions. 

\begin{key}{/tikz/data visualization/xy Cartesian}
  This axis system creates two axes called |x axis| and |y axis| that
  point right and up, respectively. By default, one unit is mapped to
  one cm.

\begin{codeexample}[]
\begin{tikzpicture}
  \datavisualization [xy Cartesian, visualize as smooth line] 
    data [format=function] {
      var x : interval [-1.25:1.25];
      func y = \value x*\value x*\value x;
    };
\end{tikzpicture}
\end{codeexample}
  
  
  \begin{key}{/tikz/data visualization/xy axes=\meta{options}}
    This key applies the \meta{options} both to the |x axis| and the
    |y axis|. 
  \end{key}

\end{key}


\begin{key}{/tikz/data visualization/xyz Cartesian cabinet}
  This axis system works like |xy Cartesian|, only it
  \emph{additionally} creates an axis called |z axis| that points left
  and down. For this axis, one unit corresponds to $\frac{1}{2}\sin
  45^\circ\mathrm{cm}$. This is also known as a cabinet projection.
  
  \begin{key}{/tikz/data visualization/xyz axes=\meta{options}}
    This key applies the \meta{options} both to the |x axis| and the
    |y axis|.   
  \end{key}

\end{key}


\begin{key}{/tikz/data visualization/uv Cartesian}
  This axis system works like |xy Cartesian|, but it introduces two
  axes called |u axis| and |v axis| rather than the |x axis| and the
  |y axis|. The idea is that in addition to a ``major''
  $xy$-coordinate system this is also a ``smaller'' or ``minor''
  coordinate system in use for depicting, say, small vectors with
  respect to this second coordinate system.
  
  \begin{key}{/tikz/data visualization/uv axes=\meta{options}}
    Applies the \meta{options} to both the |u axis| and the |y axis|.
  \end{key}

\end{key}

\begin{key}{/tikz/data visualization/uvw Cartesian cabinet}
  Like |xyz Cartesian cabinet|, but for the $uvw$-system.
  
  \begin{key}{/tikz/data visualization/uvw axes=\meta{options}}
    Like |xyz axes|.
  \end{key}
\end{key}



\subsection{Reference: Tick Placement Strategies}
\label{section-dv-tick-placement-strategies}


\subsubsection{Basic Strategies}

When the data visualization is requested to automatically determine
``good'' positions for the placement of ticks on an axis, it uses one
of several possible \emph{basic strategies}. These strategies differ
dramatically in which tick positions they will choose: For a range of
values between $5$ and $1000$, a |linear steps| strategy might place
ticks at positions $100$, $200$, through to $1000$, while an
|exponential steps| strategy would prefer the tick positions $10$,
$100$ and $1000$. The exact number and values of the tick positions
chosen by either strategy can be fine-tuned using additional options
like |step| or |about|.

\begin{key}{/tikz/data visualization/axis options/linear steps}
  This strategy placed ticks at positions that are evenly spaced by
  the current value of |step|.

  In detail, the following happens: Let $a$ be the minimum value of the
  data values along the axis and let $b$ be the maximum. Let the
  current \emph{stepping} be $s$ (the stepping is set using the |step|
  option, see below) and let the current \emph{phasing} be $p$ (set
  using the |phase|) option. Then ticks are placed all positions
  $i\cdots s + p$ that lie in the interval $[a,b]$, where $i$ ranges
  over all integers.

  The tick positions computed in the way described above are
  \emph{mayor} step positions. In addition to these, if the key
  |minor steps between steps| is set to some number $n$, then $n$ many
  minor ticks are introduced between each two mayor ticks (and also
  before and after the last mayor tick, provided the values still lie
  in the interval $[a,b]$). Note that is $n$ is $1$, then one minor tick
  will be added in the middle between any two mayor ticks. Use a value
  of $9$ (not $10$) to partition the interval between two mayor ticks
  into ten equally sized minor intervals.

\begin{codeexample}[]
\begin{tikzpicture}
  \datavisualization
    [scientific inner axes, scientific axes/width=3cm,
     x axis={ticks={step=3, minor steps between steps=2}},
     y axis={ticks={step=.36}},
     visualize as scatter] 
    data {
      x, y
      17, 30
      34, 32
    };
\end{tikzpicture}
\end{codeexample}  
\end{key}

\begin{key}{/tikz/data visualization/axis options/exponential steps}
  This strategy produces ticks at positions that are appropriate for
  logarithmic plots.

  In detail, the following happens: As for |linear steps| let numbers
  $a$, $b$, $s$, and $p$ be given. Then, mayor ticks are placed at all
  positions $10^{i\cdot s+p}$ that lie in the interval $[a,b]$ for $i
  \in \mathbb Z$.

  The minor steps are added in the same way as for |linear steps|. In
  particular, they interpolate \emph{linearly} between mayor steps.
  
\begin{codeexample}[]
\begin{tikzpicture}
  \datavisualization
    [scientific axes,
     x axis={logarithmic, exponential steps,ticks={step=1.5}},
     y axis={logarithmic, exponential steps,ticks={step=1, minor steps between steps=9}},
     visualize as scatter] 
    data {
      x, y
      1, 10
      1000, 1000000
    };
\end{tikzpicture}
\end{codeexample}  
\end{key}




\subsubsection{Explicit Step Specification}

\begin{key}{/tikz/data visualization/axis options/step=\meta{value}}
\end{key}

\begin{key}{/tikz/data visualization/axis options/minor steps between steps=\meta{number}}
\end{key}

\begin{key}{/tikz/data visualization/axis options/phase=\meta{value}}
\end{key}

  
\subsubsection{Implicit Step Specification}

\begin{key}{/tikz/data visualization/axis options/about=\meta{number}}
\end{key}

\begin{key}{/tikz/data visualization/axis options/many}
\end{key}

\begin{key}{/tikz/data visualization/axis options/some}
\end{key}

\begin{key}{/tikz/data visualization/axis options/few}
\end{key}

\begin{key}{/tikz/data visualization/axis options/none}
\end{key}

\begin{key}{/tikz/data visualization/axis options/about strategy=\meta{list}}
\end{key}

\begin{key}{/tikz/data visualization/axis options/standard about
    strategy}
  Permissible values for the basic stepping are: $1$, $2$, $2.5$, or
  $5$.
  This strategy is the default strategy.
\end{key}

\begin{key}{/tikz/data visualization/axis options/euro about strategy}
  Permissible values for the basic stepping are: $1$, $2$, or
  $5$. These are the same values as for the Euro coins, hence the
  name. 
\end{key}

\begin{key}{/tikz/data visualization/axis options/half about strategy}
  Permissible values for the basic stepping are: $1$ or $5$. Use this
  strategy if only powers of $10$ or halves thereof seem logical.
\end{key}

\begin{key}{/tikz/data visualization/axis options/quarter about strategy}
  Permissible values for the basic stepping are: $1$, $2.5$, or $5$. 
\end{key}

\begin{key}{/tikz/data visualization/axis options/int about strategy}
  Permissible values for the basic stepping are: $1$, $2$, $3$, $4$,
  or $5$.
\end{key}



\subsubsection{Advanced: Defining New Placing Strategies}

\begin{key}{/tikz/data visualization/axis options/basic strategy=\meta{macro}}
  
\end{key}
  
\subsection{Advanced: Creating New Axes}

\subsection{Advanced: Creating New Axis Systems}
