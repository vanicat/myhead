% This file has been generated from the lua sources using LuaDoc.
% To regenerate it call "make genluadoc" in
% doc/generic/pgf/version-for-luatex/en.

\begin{filedescription}{pgflibrarygraphdrawing-interface.lua}


\begin{luacommand}{{Interface:addEdge}(\meta{from},\meta{to},\meta{direction},\meta{edge\_nodes},\meta{options},\meta{tikz\_options})}
Adds an edge from one node to another by name.  Both parameters are node names and have to exist before an edge can be created between them. 

Parameters:
\begin{parameterdescription}
	\item[\meta{from}] Name of the node the edge begins at.\item[\meta{to}] Name of the node the edge ends at.\item[\meta{direction}] Direction of the edge (e.g. |--| for an undirected edge or |->| for a directed edge from the first to the second node).\item[\meta{edge\_nodes}] A string for \tikzname\ to generate the edge label nodes later. Needs to be passed back to TikZ unmodified.\item[\meta{options}] A string of |{key}{value}| pairs of edge options that are relevant to graph drawing algorithms.\item[\meta{tikz\_options}] A string of |{key}{value}| pairs that need to be passed back to \tikzname\ unmodified. 
\end{parameterdescription}



See also:
\begin{itemize}
	\item[] |addNode |
\end{itemize}

\end{luacommand}
\begin{luacommand}{{Interface:addNode}(\meta{name},\meta{xMin},\meta{yMin},\meta{xMax},\meta{yMax},\meta{options})}
Adds a new node to the graph.  The options string of |{key}{value}| pairs is parsed and assigned to the node. Graph drawing algorithms may use these options to treat the node in special ways. 

Parameters:
\begin{parameterdescription}
	\item[\meta{name}] Name of the node.\item[\meta{xMin}] Minimum x point of the bouding box.\item[\meta{yMin}] Minimum y point of the bouding box.\item[\meta{xMax}] Maximum x point of the bouding box.\item[\meta{yMax}] Maximum y point of the bouding box.\item[\meta{options}] Options for the node. 
\end{parameterdescription}



\end{luacommand}
\begin{luacommand}{{Interface:drawEdge}(\meta{edge})}
Passes an edge back to the \TeX\ layer.  Edges with a direction of |Edge.NONE| are skipped and not passed back to \TeX. 

Parameters:
\begin{parameterdescription}
	\item[\meta{edge}] The edge to pass back to the \TeX\ layer. 
\end{parameterdescription}



\end{luacommand}
\begin{luacommand}{{Interface:drawGraph}()}
Arranges the current graph using the specified algorithm.  The algorithm is derived from the graph options and is loaded on demand from the corresponding algorithm file. For a fictitious algorithm |simple| this file is per convention called |pgflibrarygraphdrawing-algorithms-simple.lua|. It is required to define at least one function as an entry point to the algorithm. The name of the function is again predetermined as |drawGraphAlgorithm_simple|. When a graph is to be layed out, this function is called with the graph as its only parameter. 



\end{luacommand}
\begin{luacommand}{{Interface:drawNode}(\meta{node})}
Passes a node back to the \TeX\ layer. 

Parameters:
\begin{parameterdescription}
	\item[\meta{node}] The node to pass back to the \TeX\ layer. 
\end{parameterdescription}



\end{luacommand}
\begin{luacommand}{{Interface:finishGraph}()}
Passes the current graph back to the \TeX\ layer and removes it from the stack. 



\end{luacommand}
\begin{luacommand}{{Interface:getOption}(\meta{name})}
Returns the value of the graph option \meta{name}. 

Parameters:
\begin{parameterdescription}
	\item[\meta{name}] Name of the option. 
\end{parameterdescription}


Return value:
\begin{parameterdescription} 
  \item[] The value of the \meta{name} option or |nil|. 
\end{parameterdescription}


\end{luacommand}
\begin{luacommand}{{Interface:loadAlgorithm}(\meta{name})}
Attempts to load the algorithm with the given \meta{name}.  This function tries to look up the corresponding algorithm file |pgflibrarygraphdrawing-algorithms-name.lua| and attempts to look up the main function for calling the algorithm. 

Parameters:
\begin{parameterdescription}
	\item[\meta{name}] Name of the algorithm. 
\end{parameterdescription}


Return value:
\begin{parameterdescription} 
  \item[] The algorithm function or nil. 
\end{parameterdescription}


\end{luacommand}
\begin{luacommand}{{Interface:newGraph}(\meta{options})}
Creates a new graph and adds it to the graph stack.  The options string consisting of |{key}{value}| pairs is parsed and assigned to the graph. These options are used to configure the different graph drawing algorithms shipped with \tikzname. 

Parameters:
\begin{parameterdescription}
	\item[\meta{options}] A string containing |{key}{value}| pairs of \tikzname\ options. 
\end{parameterdescription}



See also:
\begin{itemize}
	\item[] |finishGraph |
\end{itemize}

\end{luacommand}
\begin{luacommand}{{Interface:setOption}(\meta{name},\meta{value})}
Sets the graph option \meta{name} to \meta{value}. Only affects the current graph. 

Parameters:
\begin{parameterdescription}
	\item[\meta{name}] The name of the option to set.\item[\meta{value}] New value for the option. 
\end{parameterdescription}



\end{luacommand}

\end{filedescription}