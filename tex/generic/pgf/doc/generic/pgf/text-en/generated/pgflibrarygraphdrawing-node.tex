% This file has been generated from the lua sources using LuaDoc.
% To regenerate it call "make genluadoc" in
% doc/generic/pgf/version-for-luatex/en.

\begin{filedescription}{pgflibrarygraphdrawing-node.lua}


\begin{luacommand}{{Node:\textunderscore{}\textunderscore{}eq}(\meta{object},\meta{other})}
Compares two nodes by their name. 

Parameters:
\begin{parameterdescription}
	\item[\meta{other}] Another node to compare with. 
\end{parameterdescription}


Return value:
\begin{parameterdescription} 
  \item[] |true| if both nodes have the same name. |false| otherwise. 
\end{parameterdescription}


\end{luacommand}
\begin{luacommand}{{Node:\textunderscore{}\textunderscore{}tostring}()}
Returns a formated string representation of the node. 


Return value:
\begin{parameterdescription} 
  \item[] String represenation of the node. 
\end{parameterdescription}


\end{luacommand}
\begin{luacommand}{{Node:addEdge}(\meta{edge})}
Adds new edge to the node. 

Parameters:
\begin{parameterdescription}
	\item[\meta{edge}] The edge to be added. 
\end{parameterdescription}



\end{luacommand}
\begin{luacommand}{{Node:copy}()}
Creates a shallow copy of the node.  Most notably, the edges adjacent are not preserved in the copy. 


Return value:
\begin{parameterdescription} 
  \item[] Copy of the node. 
\end{parameterdescription}


\end{luacommand}
\begin{luacommand}{{Node:getDegree}()}
Counts the adjacent edges of the node. 


Return value:
\begin{parameterdescription} 
  \item[] The number of adjacent edges of the node. 
\end{parameterdescription}


\end{luacommand}
\begin{luacommand}{{Node:getEdges}()}
Returns all edges of the node.  Instead of calling |node:getEdges()| the edges can alternatively be accessed directly with |node.edges|. 


Return value:
\begin{parameterdescription} 
  \item[] All edges of the node. 
\end{parameterdescription}


\end{luacommand}
\begin{luacommand}{{Node:getInDegree}(\meta{ignore\_reversed})}
Returns the number of incoming edges of the node. 

Parameters:
\begin{parameterdescription}
	\item[\meta{ignore\_reversed}] Optional parameter to consider reversed edges not reversed for this method call. Defaults to |false|. 
\end{parameterdescription}


Return value:
\begin{parameterdescription} 
  \item[] The number of incoming edges of the node. 
\end{parameterdescription}


See also:
\begin{itemize}
	\item[] |Node:getIncomingEdges(reversed) |
\end{itemize}

\end{luacommand}
\begin{luacommand}{{Node:getIncomingEdges}(\meta{ignore\_reversed})}
Returns the incoming edges of the node. Undefined result for hyperedges. 

Parameters:
\begin{parameterdescription}
	\item[\meta{ignore\_reversed}] Optional parameter to consider reversed edges not reversed for this method call. Defaults to |false|. 
\end{parameterdescription}


Return value:
\begin{parameterdescription} 
  \item[] Incoming edges of the node. This includes undirected edges and directed edges pointing to the node. 
\end{parameterdescription}


\end{luacommand}
\begin{luacommand}{{Node:getOption}(\meta{name})}
Returns the value of the node option \meta{name}. 

Parameters:
\begin{parameterdescription}
	\item[\meta{name}] Name of the node option. 
\end{parameterdescription}


Return value:
\begin{parameterdescription} 
  \item[] The value of the node option \meta{name} or |nil|. 
\end{parameterdescription}


\end{luacommand}
\begin{luacommand}{{Node:getOutDegree}(\meta{ignore\_reversed})}
Returns the number of edges starting at the node. 

Parameters:
\begin{parameterdescription}
	\item[\meta{ignore\_reversed}] Optional parameter to consider reversed edges not reversed for this method call. Defaults to |false|. 
\end{parameterdescription}


Return value:
\begin{parameterdescription} 
  \item[] The number of outgoing edges of the node. 
\end{parameterdescription}


See also:
\begin{itemize}
	\item[] |Node:getOutgoingEdges() |
\end{itemize}

\end{luacommand}
\begin{luacommand}{{Node:getOutgoingEdges}(\meta{ignore\_reversed})}
Returns the outgoing edges of the node. Undefined result for hyperedges. 

Parameters:
\begin{parameterdescription}
	\item[\meta{ignore\_reversed}] Optional parameter to consider reversed edges not reversed for this method call. Defaults to |false|. 
\end{parameterdescription}


Return value:
\begin{parameterdescription} 
  \item[] Outgoing edges of the node. This includes undirected edges and directed edges leaving the node. 
\end{parameterdescription}


\end{luacommand}
\begin{luacommand}{{Node:getTexHeight}()}
Computes the heigth of the node. 


Return value:
\begin{parameterdescription} 
  \item[] Height of the node. 
\end{parameterdescription}


\end{luacommand}
\begin{luacommand}{{Node:getTexWidth}()}
Computes the width of the node. 


Return value:
\begin{parameterdescription} 
  \item[] Width of the node. 
\end{parameterdescription}


\end{luacommand}
\begin{luacommand}{{Node:new}(\meta{values})}
Creates a new node. 

Parameters:
\begin{parameterdescription}
	\item[\meta{values}] Values to override default node settings. The following parameters can be set:\par |name|: The name of the node. It is obligatory to define this.\par |tex|: Information about the corresponding \TeX\ node.\par |edges|: Edges adjacent to the node.\par |pos|: Initial position of the node.\par |options|: A table of node options passed over from \tikzname. 
\end{parameterdescription}


Return value:
\begin{parameterdescription} 
  \item[] A newly allocated node. 
\end{parameterdescription}


\end{luacommand}
\begin{luacommand}{{Node:removeEdge}(\meta{edge})}
Removes an edge from the node. 

Parameters:
\begin{parameterdescription}
	\item[\meta{edge}] The edge to remove. 
\end{parameterdescription}



\end{luacommand}
\begin{luacommand}{{Node:setOption}(\meta{name},\meta{value})}
Sets the node option \meta{name} to \meta{value}. 

Parameters:
\begin{parameterdescription}
	\item[\meta{name}] Name of the node option to be changed.\item[\meta{value}] New value for the node option \meta{name}. 
\end{parameterdescription}



\end{luacommand}

\end{filedescription}