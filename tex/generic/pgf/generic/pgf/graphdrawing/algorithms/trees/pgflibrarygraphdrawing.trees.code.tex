% Copyright 2010 by Renée Ahrens, Olof Frahm, Jens Kluttig, Matthias Schulz, Stephan Schuster
% Copyright 2011 by Till Tantau
%
% This file may be distributed and/or modified
%
% 1. under the LaTeX Project Public License and/or
% 2. under the GNU Public License.
%
% See the file doc/generic/pgf/licenses/LICENSE for more details.

\ProvidesFileRCS[v\pgfversion] $Header: /cvsroot/pgf/pgf/generic/pgf/graphdrawing/algorithms/trees/pgflibrarygraphdrawing.trees.code.tex,v 1.3 2011/05/11 15:02:32 tantau Exp $



%
% Common tree options
%


% Declare a root
%
% Description:
%
% Normally, the first node encountered in a graph drawing scope is
% used as the root of the tree. However, you can also pass the "root"
% key to a node to specify that this node should actually be used as
% the root of the tree.
% 
% Example:
% 
% \graph [some tree layout] {
%   a -- b [root] -- c
% };

\pgfgddeclareforwardedkeys{/graph drawing}{root/.node parameter}



% Distances
% 
% Description:
% 
% These keys work as in the classical typesetting of tikz for trees. 

\pgfgddeclareforwardedkeys{/graph drawing}{
  level distance/.graph parameter=evaluate math expression,
  level distance/.parameter initial=1cm,
  %
  sibling distance/.graph parameter=evaluate math expression,
  sibling distance/.parameter initial=1cm,
}




% 
% Tree layouts
% 

\pgfgddeclareforwardedkeys{/graph drawing}{
  tree/.forward to={/graph drawing/\pgfkeysvalueof{/graph drawing/tree/default algorithm}}
}

\pgfkeys{/graph drawing/tree/default algorithm/.initial=AhrensFKSS2011 tree}



% AhrensFKSS2011 tree layout
%
% Description:
%
% This is an example tree layout algorithm an input tree by the people
% who first created the graph drawing library.
%
% Example:
%
% \tikz \graph[AhrensFKSS2011 tree] { a -> {b, c -> {d,e}}};

\pgfgddeclarealgorithmkey
{AhrensFKSS2011 tree}
{AhrensFKSS2011 tree}
{algorithm=AhrensFKSS2011-tree}




\endinput